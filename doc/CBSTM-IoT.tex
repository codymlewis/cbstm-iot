\documentclass{article}
\usepackage{graphicx}
\usepackage[margin=2cm]{geometry}
\usepackage[english]{babel}
\usepackage{float}
\usepackage{amsmath}
\usepackage{cite}
\usepackage{amssymb}
\usepackage{booktabs}
\usepackage{tabularx}
\usepackage{hyperref}
\usepackage{multicol}
\usepackage{parskip}
\title{Context Attacks on CBSTM-IoT}
\author{Cody Lewis}
\date{\today}

\begin{document}
    \maketitle

    We implemented a simulation of the trust model proposed in \cite{rafey2016cbstm},
    this trust model managed to mitigate the context attacks.

    \section{Configurations of the Simulation}
    Our simulation contained 100 nodes where 30 were adversaries, that is, to
    show the effects of the attack while still below the Byzantine threshold
    \cite{lamport2019byzantine}. Friendships between nodes was determined
    randomly such that the average amount of friends that each node had was
    appoximately 50, then $frlow$ and $frhigh$ were calculated as a standard
    deviation below and above that mean respectively.

    We defined the relationship factor with the following function,
    \begin{equation}
        R = \frac{1}{2^n}
    \end{equation}
    where $n$ is the degree of seperation between nodes $i$ and $k$. Where
    nodes belonging to the same owner have $0$ degrees of separation, and
    those belonging to different owners have $1$ degree of separation.
    Owners have at most 7 nodes belonging to them.

    We set the computing power, $CP$, to $1$ in order to maximize malevolent abilities of
    the adversaries, also to show this trust model's greatest amount of resistance
    against the attacks.

    We decided that in the case where the sum of direct trust and indirect trust was greater
    than 1, that only direct trust was used, resulting in following trust weight factors
    to be calculated as,
    \begin{equation}
        \alpha = \begin{cases}
            1 & : DT + IndT > 1 \\
            \frac{DT}{DT + IndT} & : 0 < DT + IndT \leq 1 \\
            0 & : DT + IndT = 0
        \end{cases}
    \end{equation}
    and,
    \begin{equation}
        \beta = 1 - \alpha
    \end{equation}

    The effects of this can be seen Figure \ref{fig:trust_plot}, where after appoximately 20
    transactions, the trust jumps from using a combination of direct and indirect to
    only using direct.

    Nodes in this simulation were implemented to act benevolent with any context
    value less than or equal to the their randomly assigned context potential,
    under contexts above that they will act malevolent. A context, $c_i$ is less than or
    equal to another context, $c_j$, iff $i \leq j$. The context setting
    adversaries attack by always reporting to the other nodes with a bad
    mouthed recommendation with the target context.

    \begin{table}
        \begin{tabularx}{\textwidth}{X X}
            \toprule
            \textbf{Parameter} & \textbf{Value} \\
            \midrule
            $w$ & $\{0.25, 0.25, 0.25, 0.25\}$ \\
            \midrule
            $w_h$ & $0.50$ \\
            \midrule
            $w_d$ & $0.50$ \\
            \midrule
            $R$ & $0.5$ \\
            \midrule
            $CP$ & $1$ \\
            \midrule
            $fb_{max}$ & $10$ \\
            \midrule
            $MD$ & $0.1$ \\
            \midrule
            Contexts & $\{c_1, c_2, c_3, c_4, c_5, c_6, c_7, c_8, c_9, c_{10}\}$ \\
            \midrule
            Total nodes & $ 100 $ \\
            \midrule
            Transactions & $ 50 $ \\
            \midrule
            Adversaries & $ 30\% $ \\
            \bottomrule
        \end{tabularx}
        \caption{The parameters of CBSTM IoT trust model implementation}
        \label{table:sim-params}
    \end{table}

    \newpage

    \section{Results}
    This trust model was resistant to the context attacks, due to the split of direct and
    indirect trusts, where direct trust is calculated as an agregation of
    experiences in all contexts. Another factor that allowed for the mitigation
    of context attacks was the filtration of recommendations based on context,
    where the contexts themselves do not have any other effect on resulting
    trust calculation. Since the contexts were only used for the filtration of
    recommendations, standard bad mouthing mitigation techniques were effective
    against the context setting attack. The comparison of the context setting
    attack to a completely benevolent network is shown in Figure \ref{fig:trust_plot}, where the red
    line is the trust values where 30\% of network are context setting adversaries, and
    the blue line is when there are no adversaries in the network.

    \begin{figure}[H]
        \centering
        \includegraphics[width=\textwidth]{../1-trust-eval-on-8-in-context-3.png}
        \caption{Comparison of the context setting attack to a benevolent network}
        \label{fig:trust_plot}
    \end{figure}


    \bibliographystyle{plain}
    \bibliography{CBSTM-IoT}
\end{document}
