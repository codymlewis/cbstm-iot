\documentclass{article}
\usepackage{graphicx}
\usepackage[margin=2cm]{geometry}
\usepackage[english]{babel}
\usepackage{float}
\usepackage{amsmath}
\usepackage{cite}
\usepackage{amssymb}
\usepackage{booktabs}
\usepackage{tabularx}
\usepackage{hyperref}
\usepackage{multicol}
\usepackage{parskip}
\title{Context Attacks on CBSM-IoT}
\author{Cody Lewis}
\date{\today}

\begin{document}
    \maketitle

    Our simulation contained 100 nodes where 30 were adversaries, that is, to
    show the effects of the attack while still below the Byzantine threshold.

    Let $frlow = 30$ and $frhigh = 70$.

    Define the relationship factor with the following function,
    \begin{equation}
        R = \frac{1}{2^n}
    \end{equation}
    where $n$ is the degree of seperation between nodes $i$ and $k$.

    Set the computing power, $CP$, to $1$ in order to maximize malevolent abilities of
    the adversaries, also to show this trust model's greatest amount of resistance
    against the attacks.

    \bibliographystyle{plain}
    \bibliography{CBSM-IoT}
\end{document}
